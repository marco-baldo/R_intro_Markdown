% Options for packages loaded elsewhere
\PassOptionsToPackage{unicode}{hyperref}
\PassOptionsToPackage{hyphens}{url}
%
\documentclass[
]{article}
\usepackage{amsmath,amssymb}
\usepackage{lmodern}
\usepackage{iftex}
\ifPDFTeX
  \usepackage[T1]{fontenc}
  \usepackage[utf8]{inputenc}
  \usepackage{textcomp} % provide euro and other symbols
\else % if luatex or xetex
  \usepackage{unicode-math}
  \defaultfontfeatures{Scale=MatchLowercase}
  \defaultfontfeatures[\rmfamily]{Ligatures=TeX,Scale=1}
\fi
% Use upquote if available, for straight quotes in verbatim environments
\IfFileExists{upquote.sty}{\usepackage{upquote}}{}
\IfFileExists{microtype.sty}{% use microtype if available
  \usepackage[]{microtype}
  \UseMicrotypeSet[protrusion]{basicmath} % disable protrusion for tt fonts
}{}
\makeatletter
\@ifundefined{KOMAClassName}{% if non-KOMA class
  \IfFileExists{parskip.sty}{%
    \usepackage{parskip}
  }{% else
    \setlength{\parindent}{0pt}
    \setlength{\parskip}{6pt plus 2pt minus 1pt}}
}{% if KOMA class
  \KOMAoptions{parskip=half}}
\makeatother
\usepackage{xcolor}
\usepackage[margin=1in]{geometry}
\usepackage{color}
\usepackage{fancyvrb}
\newcommand{\VerbBar}{|}
\newcommand{\VERB}{\Verb[commandchars=\\\{\}]}
\DefineVerbatimEnvironment{Highlighting}{Verbatim}{commandchars=\\\{\}}
% Add ',fontsize=\small' for more characters per line
\usepackage{framed}
\definecolor{shadecolor}{RGB}{248,248,248}
\newenvironment{Shaded}{\begin{snugshade}}{\end{snugshade}}
\newcommand{\AlertTok}[1]{\textcolor[rgb]{0.94,0.16,0.16}{#1}}
\newcommand{\AnnotationTok}[1]{\textcolor[rgb]{0.56,0.35,0.01}{\textbf{\textit{#1}}}}
\newcommand{\AttributeTok}[1]{\textcolor[rgb]{0.77,0.63,0.00}{#1}}
\newcommand{\BaseNTok}[1]{\textcolor[rgb]{0.00,0.00,0.81}{#1}}
\newcommand{\BuiltInTok}[1]{#1}
\newcommand{\CharTok}[1]{\textcolor[rgb]{0.31,0.60,0.02}{#1}}
\newcommand{\CommentTok}[1]{\textcolor[rgb]{0.56,0.35,0.01}{\textit{#1}}}
\newcommand{\CommentVarTok}[1]{\textcolor[rgb]{0.56,0.35,0.01}{\textbf{\textit{#1}}}}
\newcommand{\ConstantTok}[1]{\textcolor[rgb]{0.00,0.00,0.00}{#1}}
\newcommand{\ControlFlowTok}[1]{\textcolor[rgb]{0.13,0.29,0.53}{\textbf{#1}}}
\newcommand{\DataTypeTok}[1]{\textcolor[rgb]{0.13,0.29,0.53}{#1}}
\newcommand{\DecValTok}[1]{\textcolor[rgb]{0.00,0.00,0.81}{#1}}
\newcommand{\DocumentationTok}[1]{\textcolor[rgb]{0.56,0.35,0.01}{\textbf{\textit{#1}}}}
\newcommand{\ErrorTok}[1]{\textcolor[rgb]{0.64,0.00,0.00}{\textbf{#1}}}
\newcommand{\ExtensionTok}[1]{#1}
\newcommand{\FloatTok}[1]{\textcolor[rgb]{0.00,0.00,0.81}{#1}}
\newcommand{\FunctionTok}[1]{\textcolor[rgb]{0.00,0.00,0.00}{#1}}
\newcommand{\ImportTok}[1]{#1}
\newcommand{\InformationTok}[1]{\textcolor[rgb]{0.56,0.35,0.01}{\textbf{\textit{#1}}}}
\newcommand{\KeywordTok}[1]{\textcolor[rgb]{0.13,0.29,0.53}{\textbf{#1}}}
\newcommand{\NormalTok}[1]{#1}
\newcommand{\OperatorTok}[1]{\textcolor[rgb]{0.81,0.36,0.00}{\textbf{#1}}}
\newcommand{\OtherTok}[1]{\textcolor[rgb]{0.56,0.35,0.01}{#1}}
\newcommand{\PreprocessorTok}[1]{\textcolor[rgb]{0.56,0.35,0.01}{\textit{#1}}}
\newcommand{\RegionMarkerTok}[1]{#1}
\newcommand{\SpecialCharTok}[1]{\textcolor[rgb]{0.00,0.00,0.00}{#1}}
\newcommand{\SpecialStringTok}[1]{\textcolor[rgb]{0.31,0.60,0.02}{#1}}
\newcommand{\StringTok}[1]{\textcolor[rgb]{0.31,0.60,0.02}{#1}}
\newcommand{\VariableTok}[1]{\textcolor[rgb]{0.00,0.00,0.00}{#1}}
\newcommand{\VerbatimStringTok}[1]{\textcolor[rgb]{0.31,0.60,0.02}{#1}}
\newcommand{\WarningTok}[1]{\textcolor[rgb]{0.56,0.35,0.01}{\textbf{\textit{#1}}}}
\usepackage{longtable,booktabs,array}
\usepackage{calc} % for calculating minipage widths
% Correct order of tables after \paragraph or \subparagraph
\usepackage{etoolbox}
\makeatletter
\patchcmd\longtable{\par}{\if@noskipsec\mbox{}\fi\par}{}{}
\makeatother
% Allow footnotes in longtable head/foot
\IfFileExists{footnotehyper.sty}{\usepackage{footnotehyper}}{\usepackage{footnote}}
\makesavenoteenv{longtable}
\usepackage{graphicx}
\makeatletter
\def\maxwidth{\ifdim\Gin@nat@width>\linewidth\linewidth\else\Gin@nat@width\fi}
\def\maxheight{\ifdim\Gin@nat@height>\textheight\textheight\else\Gin@nat@height\fi}
\makeatother
% Scale images if necessary, so that they will not overflow the page
% margins by default, and it is still possible to overwrite the defaults
% using explicit options in \includegraphics[width, height, ...]{}
\setkeys{Gin}{width=\maxwidth,height=\maxheight,keepaspectratio}
% Set default figure placement to htbp
\makeatletter
\def\fps@figure{htbp}
\makeatother
\setlength{\emergencystretch}{3em} % prevent overfull lines
\providecommand{\tightlist}{%
  \setlength{\itemsep}{0pt}\setlength{\parskip}{0pt}}
\setcounter{secnumdepth}{-\maxdimen} % remove section numbering
\ifLuaTeX
  \usepackage{selnolig}  % disable illegal ligatures
\fi
\IfFileExists{bookmark.sty}{\usepackage{bookmark}}{\usepackage{hyperref}}
\IfFileExists{xurl.sty}{\usepackage{xurl}}{} % add URL line breaks if available
\urlstyle{same} % disable monospaced font for URLs
\hypersetup{
  pdftitle={Radar Chart in Markdown},
  pdfauthor={Marco},
  hidelinks,
  pdfcreator={LaTeX via pandoc}}

\title{Radar Chart in Markdown}
\author{Marco}
\date{2023-06-19}

\begin{document}
\maketitle

\hypertarget{script-for-a-radar-chart-example-with-r}{%
\section{Script for a Radar Chart example with
R}\label{script-for-a-radar-chart-example-with-r}}

\begin{longtable}[]{@{}
  >{\raggedright\arraybackslash}p{(\columnwidth - 0\tabcolsep) * \real{0.6389}}@{}}
\toprule()
\endhead
Markdown is an \textbf{easy to use} format for writing reports. It
resembles what you naturally write every time you compose an email. In
fact, you may have already used markdown \emph{without realizing it}.
These websites all rely on markdown formatting. In case word is not
working use \emph{tinytex::install\_tinytex()} before to start. \\
* \href{www.github.com}{Github} \\
* \href{www.stackoverflow.com}{StackOverflow} \\
* \href{www.reddit.com}{Reddit} \\
eval = TRUE evaluate the script \\
echo = FALSE means only the result shown \\
message = FALSE remove the messages from the report \\
warnings = FALSE remove the warning messages \\
\bottomrule()
\end{longtable}

\hypertarget{r-markdown}{%
\subsection{R Markdown}\label{r-markdown}}

This is an R Markdown document. Markdown is a simple formatting syntax
for authoring HTML, PDF, and MS Word documents. For more details on
using R Markdown see \url{http://rmarkdown.rstudio.com}.

When you click the \textbf{Knit} button a document will be generated
that includes both content as well as the output of any embedded R code
chunks within the document. You can embed an R code chunk like this:

\hypertarget{create-data---to-let-the-radar-chart-work-properly-you-need-to-specify-maximum-minumum-and-the-value-of-the-variable.-for-example}{%
\section{\texorpdfstring{Create Data - To let the radar chart work
properly you need to specify maximum, minumum and the value of the
variable. \href{https://www.statology.org/radar-chart-in-r/}{For
example}:}{Create Data - To let the radar chart work properly you need to specify maximum, minumum and the value of the variable. For example:}}\label{create-data---to-let-the-radar-chart-work-properly-you-need-to-specify-maximum-minumum-and-the-value-of-the-variable.-for-example}}

\begin{Shaded}
\begin{Highlighting}[]
\NormalTok{df }\OtherTok{\textless{}{-}} \FunctionTok{data.frame}\NormalTok{(}\AttributeTok{Mon=}\FunctionTok{c}\NormalTok{(}\DecValTok{100}\NormalTok{, }\DecValTok{0}\NormalTok{, }\DecValTok{34}\NormalTok{),}
                 \AttributeTok{Tue=}\FunctionTok{c}\NormalTok{(}\DecValTok{100}\NormalTok{, }\DecValTok{0}\NormalTok{, }\DecValTok{48}\NormalTok{),}
                 \AttributeTok{Wed=}\FunctionTok{c}\NormalTok{(}\DecValTok{100}\NormalTok{, }\DecValTok{0}\NormalTok{, }\DecValTok{58}\NormalTok{),}
                 \AttributeTok{Thu=}\FunctionTok{c}\NormalTok{(}\DecValTok{100}\NormalTok{, }\DecValTok{0}\NormalTok{, }\DecValTok{67}\NormalTok{),}
                 \AttributeTok{Fri=}\FunctionTok{c}\NormalTok{(}\DecValTok{100}\NormalTok{, }\DecValTok{0}\NormalTok{, }\DecValTok{55}\NormalTok{),}
                 \AttributeTok{Sat=}\FunctionTok{c}\NormalTok{(}\DecValTok{100}\NormalTok{, }\DecValTok{0}\NormalTok{, }\DecValTok{29}\NormalTok{),}
                 \AttributeTok{Sun=}\FunctionTok{c}\NormalTok{(}\DecValTok{100}\NormalTok{, }\DecValTok{0}\NormalTok{, }\DecValTok{18}\NormalTok{))}

\CommentTok{\#view data}
\NormalTok{df}
\end{Highlighting}
\end{Shaded}

\begin{verbatim}
##   Mon Tue Wed Thu Fri Sat Sun
## 1 100 100 100 100 100 100 100
## 2   0   0   0   0   0   0   0
## 3  34  48  58  67  55  29  18
\end{verbatim}

\begin{Shaded}
\begin{Highlighting}[]
\CommentTok{\# Once the data is in this format, we can use the radarchart() function from the }
\CommentTok{\# fmsb library to create a basic radar chart:}


\FunctionTok{radarchart}\NormalTok{(df)}
\end{Highlighting}
\end{Shaded}

\includegraphics{Radar_chart_markdown_files/figure-latex/unnamed-chunk-1-1.pdf}

Customizing Radar Charts in R

We can customize the radar chart by using the following arguments:

\begin{itemize}
\tightlist
\item
  pcol: Line color
\item
  pfcol: Fill color
\item
  plwd: Line width
\item
  cglcol: Net color
\item
  cglty: Net line type
\item
  axislabcol: Axis label color
\item
  caxislabels: Vector of axis labels to display
\item
  cglwd: Net width
\item
  vlcex: Group labels size
\end{itemize}

Here an example from the previous data frame (NB for radar chart you
need always the data.frame format):

\begin{Shaded}
\begin{Highlighting}[]
\FunctionTok{radarchart}\NormalTok{(df,}
           \AttributeTok{axistype=}\DecValTok{1}\NormalTok{, }
           \AttributeTok{pcol=}\StringTok{\textquotesingle{}pink\textquotesingle{}}\NormalTok{,}
           \AttributeTok{pfcol=}\FunctionTok{rgb}\NormalTok{(}\FloatTok{0.9}\NormalTok{,}\FloatTok{0.2}\NormalTok{,}\FloatTok{0.5}\NormalTok{,}\FloatTok{0.3}\NormalTok{),}
           \AttributeTok{plwd=}\DecValTok{3}\NormalTok{, }
           \AttributeTok{cglcol=}\StringTok{\textquotesingle{}grey\textquotesingle{}}\NormalTok{,}
           \AttributeTok{cglty=}\DecValTok{1}\NormalTok{,}
           \AttributeTok{axislabcol=}\StringTok{\textquotesingle{}grey\textquotesingle{}}\NormalTok{,}
           \AttributeTok{cglwd=}\FloatTok{0.6}\NormalTok{,}
           \AttributeTok{vlcex=}\FloatTok{1.1}\NormalTok{,}
           \AttributeTok{title=}\StringTok{\textquotesingle{}Customers per Day\textquotesingle{}}
\NormalTok{)}
\end{Highlighting}
\end{Shaded}

\includegraphics{Radar_chart_markdown_files/figure-latex/unnamed-chunk-2-1.pdf}

\hypertarget{second-part}{%
\section{Second Part}\label{second-part}}

Doing the same but with an average or threshold line - can be useful
also for other analysis
\href{https://stackoverflow.com/questions/50353923/generate-radar-charts-with-ggplot2}{Example}

\begin{Shaded}
\begin{Highlighting}[]
\NormalTok{data }\OtherTok{\textless{}{-}} \FunctionTok{read.csv}\NormalTok{(}\StringTok{"C:/\_R/R\_visual\_radar/Radar\_chart/data.csv"}\NormalTok{)}
\FunctionTok{str}\NormalTok{(data)}
\end{Highlighting}
\end{Shaded}

\begin{verbatim}
## 'data.frame':    30 obs. of  5 variables:
##  $ X    : chr  "Mean_DC1_SAPvsSH" "Mean_DC1_SAPvsTD6" "Mean_DC1_SHvsTD6" "Mean_DC2_SAPvsSH" ...
##  $ Count: num  -15.26 NaN NaN -7.17 -49.37 ...
##  $ X1   : chr  "Mean" "Mean" "Mean" "Mean" ...
##  $ X2   : chr  "DC1" "DC1" "DC1" "DC2" ...
##  $ X3   : chr  "SAPvsSH" "SAPvsTD6" "SHvsTD6" "SAPvsSH" ...
\end{verbatim}

\begin{Shaded}
\begin{Highlighting}[]
\CommentTok{\# data \textless{}{-} data[,2:5]}
\CommentTok{\# str(data)}


\FunctionTok{ggplot}\NormalTok{(}\AttributeTok{data=}\NormalTok{data,  }\FunctionTok{aes}\NormalTok{(}\AttributeTok{x=}\NormalTok{X2, }\AttributeTok{y=}\NormalTok{Count, }\AttributeTok{group=}\NormalTok{X3, }\AttributeTok{colour=}\NormalTok{X3)) }\SpecialCharTok{+} 
  \FunctionTok{geom\_point}\NormalTok{(}\AttributeTok{size=}\DecValTok{5}\NormalTok{) }\SpecialCharTok{+} 
  \FunctionTok{geom\_line}\NormalTok{() }\SpecialCharTok{+} 
  \FunctionTok{xlab}\NormalTok{(}\StringTok{"Decils"}\NormalTok{) }\SpecialCharTok{+} 
  \FunctionTok{ylab}\NormalTok{(}\StringTok{"\% difference in nº Pk"}\NormalTok{) }\SpecialCharTok{+} 
  \FunctionTok{ylim}\NormalTok{(}\SpecialCharTok{{-}}\DecValTok{50}\NormalTok{,}\DecValTok{25}\NormalTok{) }\SpecialCharTok{+} \FunctionTok{ggtitle}\NormalTok{(}\StringTok{"CL"}\NormalTok{)  }\SpecialCharTok{+} 
  \FunctionTok{geom\_hline}\NormalTok{(}\FunctionTok{aes}\NormalTok{(}\AttributeTok{yintercept=}\DecValTok{0}\NormalTok{), }\AttributeTok{lwd=}\DecValTok{1}\NormalTok{, }\AttributeTok{lty=}\DecValTok{2}\NormalTok{) }\SpecialCharTok{+} 
  \FunctionTok{scale\_x\_discrete}\NormalTok{(}\AttributeTok{limits=}\FunctionTok{c}\NormalTok{(}\StringTok{"DC1"}\NormalTok{,}\StringTok{"DC2"}\NormalTok{,}\StringTok{"DC3"}\NormalTok{,}\StringTok{"DC4"}\NormalTok{,}\StringTok{"DC5"}\NormalTok{,}\StringTok{"DC6"}\NormalTok{,}\StringTok{"DC7"}\NormalTok{,}\StringTok{"DC8"}\NormalTok{,}\StringTok{"DC9"}\NormalTok{,}\StringTok{"DC10"}\NormalTok{))}
\end{Highlighting}
\end{Shaded}

\includegraphics{Radar_chart_markdown_files/figure-latex/unnamed-chunk-3-1.pdf}

I would like to transform this chart in a radar chart. I tried to use
ggradar or ggRadar, but unsuccessfully. Something like this would be
amazing:

\begin{Shaded}
\begin{Highlighting}[]
\FunctionTok{ggplot}\NormalTok{(}\AttributeTok{data=}\NormalTok{data,  }\FunctionTok{aes}\NormalTok{(}\AttributeTok{x=}\NormalTok{X2, }\AttributeTok{y=}\NormalTok{Count, }\AttributeTok{group=}\NormalTok{X3, }\AttributeTok{colour=}\NormalTok{X3)) }\SpecialCharTok{+} 
  \FunctionTok{geom\_point}\NormalTok{(}\AttributeTok{size=}\DecValTok{5}\NormalTok{) }\SpecialCharTok{+} 
  \FunctionTok{geom\_line}\NormalTok{() }\SpecialCharTok{+} 
  \FunctionTok{xlab}\NormalTok{(}\StringTok{"Decils"}\NormalTok{) }\SpecialCharTok{+} 
  \FunctionTok{ylab}\NormalTok{(}\StringTok{"\% difference in nº Pk"}\NormalTok{) }\SpecialCharTok{+} 
  \FunctionTok{ylim}\NormalTok{(}\SpecialCharTok{{-}}\DecValTok{50}\NormalTok{,}\DecValTok{25}\NormalTok{) }\SpecialCharTok{+} \FunctionTok{ggtitle}\NormalTok{(}\StringTok{"CL"}\NormalTok{)  }\SpecialCharTok{+} 
  \FunctionTok{geom\_hline}\NormalTok{(}\FunctionTok{aes}\NormalTok{(}\AttributeTok{yintercept=}\DecValTok{0}\NormalTok{), }\AttributeTok{lwd=}\DecValTok{1}\NormalTok{, }\AttributeTok{lty=}\DecValTok{2}\NormalTok{) }\SpecialCharTok{+} 
  \FunctionTok{scale\_x\_discrete}\NormalTok{(}\AttributeTok{limits=}\FunctionTok{c}\NormalTok{(}\StringTok{"DC1"}\NormalTok{,}\StringTok{"DC2"}\NormalTok{,}\StringTok{"DC3"}\NormalTok{,}\StringTok{"DC4"}\NormalTok{,}\StringTok{"DC5"}\NormalTok{,}\StringTok{"DC6"}\NormalTok{,}\StringTok{"DC7"}\NormalTok{,}\StringTok{"DC8"}\NormalTok{,}\StringTok{"DC9"}\NormalTok{,}\StringTok{"DC10"}\NormalTok{)) }\SpecialCharTok{+}
  \FunctionTok{coord\_polar}\NormalTok{()}
\end{Highlighting}
\end{Shaded}

\includegraphics{Radar_chart_markdown_files/figure-latex/unnamed-chunk-4-1.pdf}

\hypertarget{third-part---the-most-important-10}{%
\section{Third Part - The most important
10+}\label{third-part---the-most-important-10}}

Followed by this
\href{https://www.datanovia.com/en/blog/beautiful-radar-chart-in-r-using-fmsb-and-ggplot-packages/}{website}
Beautiful Radar Chart in R using FMSB and GGPlot Packages

A radar chart, also known as a spider plot is used to visualize the
values or scores assigned to an individual over multiple quantitative
variables, where each variable corresponds to a specific axis.

This article describes how to create a radar chart in R using two
different packages: the fmsb or the ggradar R packages.

Note that, the fmsb radar chart is an R base plot. The ggradar package
builds a ggplot spider plot.

You will learn:

\begin{itemize}
\tightlist
\item
  how to create a beautiful fmsb radar chart
\item
  how to create ggplot radar chart
\item
  alternatives to radar charts
\end{itemize}

\textbf{Demo data}

We'll use a demo data containing exam scores for 3 students on 9 topics
(Biology, Physics, etc). The scores range from 0 to 20. Columns are
quantitative variables and rows are individuals:

\begin{Shaded}
\begin{Highlighting}[]
\CommentTok{\# Demo data}
\NormalTok{exam\_scores }\OtherTok{\textless{}{-}} \FunctionTok{data.frame}\NormalTok{(}
  \AttributeTok{row.names =} \FunctionTok{c}\NormalTok{(}\StringTok{"Student.1"}\NormalTok{, }\StringTok{"Student.2"}\NormalTok{, }\StringTok{"Student.3"}\NormalTok{),}
  \AttributeTok{Biology =} \FunctionTok{c}\NormalTok{(}\FloatTok{7.9}\NormalTok{, }\FloatTok{3.9}\NormalTok{, }\FloatTok{9.4}\NormalTok{),}
  \AttributeTok{Physics =} \FunctionTok{c}\NormalTok{(}\DecValTok{10}\NormalTok{, }\DecValTok{20}\NormalTok{, }\DecValTok{0}\NormalTok{),}
  \AttributeTok{Maths =} \FunctionTok{c}\NormalTok{(}\FloatTok{3.7}\NormalTok{, }\FloatTok{11.5}\NormalTok{, }\FloatTok{2.5}\NormalTok{),}
  \AttributeTok{Sport =} \FunctionTok{c}\NormalTok{(}\FloatTok{8.7}\NormalTok{, }\DecValTok{20}\NormalTok{, }\DecValTok{4}\NormalTok{),}
  \AttributeTok{English =} \FunctionTok{c}\NormalTok{(}\FloatTok{7.9}\NormalTok{, }\FloatTok{7.2}\NormalTok{, }\FloatTok{12.4}\NormalTok{),}
  \AttributeTok{Geography =} \FunctionTok{c}\NormalTok{(}\FloatTok{6.4}\NormalTok{, }\FloatTok{10.5}\NormalTok{, }\FloatTok{6.5}\NormalTok{),}
  \AttributeTok{Art =} \FunctionTok{c}\NormalTok{(}\FloatTok{2.4}\NormalTok{, }\FloatTok{0.2}\NormalTok{, }\FloatTok{9.8}\NormalTok{),}
  \AttributeTok{Programming =} \FunctionTok{c}\NormalTok{(}\DecValTok{0}\NormalTok{, }\DecValTok{0}\NormalTok{, }\DecValTok{20}\NormalTok{),}
  \AttributeTok{Music =} \FunctionTok{c}\NormalTok{(}\DecValTok{20}\NormalTok{, }\DecValTok{20}\NormalTok{, }\DecValTok{20}\NormalTok{)}
\NormalTok{)}
\NormalTok{exam\_scores}
\end{Highlighting}
\end{Shaded}

\begin{verbatim}
##           Biology Physics Maths Sport English Geography Art Programming Music
## Student.1     7.9      10   3.7   8.7     7.9       6.4 2.4           0    20
## Student.2     3.9      20  11.5  20.0     7.2      10.5 0.2           0    20
## Student.3     9.4       0   2.5   4.0    12.4       6.5 9.8          20    20
\end{verbatim}

\textbf{Data preparation}

The data should be organized as follow:

\begin{itemize}
\tightlist
\item
  \textbf{The row 1 must contain the maximum values for each variable}
\item
  \textbf{The row 2 must contain the minimum values for each variable}
\item
  \textbf{Data for cases or individuals should be given starting from
  row 3}
\item
  \textbf{The number of columns or variables must be more than 2.}
\end{itemize}

\begin{Shaded}
\begin{Highlighting}[]
\CommentTok{\# Define the variable ranges: maximum and minimum}
\NormalTok{max\_min }\OtherTok{\textless{}{-}} \FunctionTok{data.frame}\NormalTok{(}
  \AttributeTok{Biology =} \FunctionTok{c}\NormalTok{(}\DecValTok{20}\NormalTok{, }\DecValTok{0}\NormalTok{), }\AttributeTok{Physics =} \FunctionTok{c}\NormalTok{(}\DecValTok{20}\NormalTok{, }\DecValTok{0}\NormalTok{), }\AttributeTok{Maths =} \FunctionTok{c}\NormalTok{(}\DecValTok{20}\NormalTok{, }\DecValTok{0}\NormalTok{),}
  \AttributeTok{Sport =} \FunctionTok{c}\NormalTok{(}\DecValTok{20}\NormalTok{, }\DecValTok{0}\NormalTok{), }\AttributeTok{English =} \FunctionTok{c}\NormalTok{(}\DecValTok{20}\NormalTok{, }\DecValTok{0}\NormalTok{), }\AttributeTok{Geography =} \FunctionTok{c}\NormalTok{(}\DecValTok{20}\NormalTok{, }\DecValTok{0}\NormalTok{),}
  \AttributeTok{Art =} \FunctionTok{c}\NormalTok{(}\DecValTok{20}\NormalTok{, }\DecValTok{0}\NormalTok{), }\AttributeTok{Programming =} \FunctionTok{c}\NormalTok{(}\DecValTok{20}\NormalTok{, }\DecValTok{0}\NormalTok{), }\AttributeTok{Music =} \FunctionTok{c}\NormalTok{(}\DecValTok{20}\NormalTok{, }\DecValTok{0}\NormalTok{)}
\NormalTok{)}
\FunctionTok{rownames}\NormalTok{(max\_min) }\OtherTok{\textless{}{-}} \FunctionTok{c}\NormalTok{(}\StringTok{"Max"}\NormalTok{, }\StringTok{"Min"}\NormalTok{)}

\CommentTok{\# Bind the variable ranges to the data}
\NormalTok{df }\OtherTok{\textless{}{-}} \FunctionTok{rbind}\NormalTok{(max\_min, exam\_scores)}
\NormalTok{df}
\end{Highlighting}
\end{Shaded}

\begin{verbatim}
##           Biology Physics Maths Sport English Geography  Art Programming Music
## Max          20.0      20  20.0  20.0    20.0      20.0 20.0          20    20
## Min           0.0       0   0.0   0.0     0.0       0.0  0.0           0     0
## Student.1     7.9      10   3.7   8.7     7.9       6.4  2.4           0    20
## Student.2     3.9      20  11.5  20.0     7.2      10.5  0.2           0    20
## Student.3     9.4       0   2.5   4.0    12.4       6.5  9.8          20    20
\end{verbatim}

\textbf{Basic radar plot}

\begin{Shaded}
\begin{Highlighting}[]
\CommentTok{\# Plot the data for student 1}
\FunctionTok{library}\NormalTok{(fmsb)}
\NormalTok{student1\_data }\OtherTok{\textless{}{-}}\NormalTok{ df[}\FunctionTok{c}\NormalTok{(}\StringTok{"Max"}\NormalTok{, }\StringTok{"Min"}\NormalTok{, }\StringTok{"Student.1"}\NormalTok{), ]}
\FunctionTok{radarchart}\NormalTok{(student1\_data)}
\end{Highlighting}
\end{Shaded}

\includegraphics{Radar_chart_markdown_files/figure-latex/unnamed-chunk-7-1.pdf}

\hypertarget{customize-the-radar-charts}{%
\section{Customize the radar charts}\label{customize-the-radar-charts}}

Key arguments to customize the different components of the fmsb radar
chart:

\begin{itemize}
\item
  \textbf{Variable options}
\item
\begin{verbatim}
   *vlabels: variable labels*
\end{verbatim}
\item
\begin{verbatim}
   *vlcex: controls the font size of variable labels*
\end{verbatim}
\item
  \textbf{Polygon options:}
\item
\begin{verbatim}
 *pcol: line color*
\end{verbatim}
\item
\begin{verbatim}
 *pfcol: fill color*
\end{verbatim}
\item
\begin{verbatim}
 *plwd: line width*
\end{verbatim}
\item
\begin{verbatim}
 *plty: line types. Can be a numeric vector 1:6 or a character vector c(“solid”, “dashed”, “dotted”, “dotdash”, “longdash”, “twodash”). To remove the line, use plty = 0 or plty = “blank”.*
\end{verbatim}
\item
  \textbf{Grid options:}
\item
\begin{verbatim}
   *cglcol: line color*
\end{verbatim}
\item
\begin{verbatim}
   *cglty: line type*
\end{verbatim}
\item
\begin{verbatim}
   *cglwd: line width*
\end{verbatim}
\item
  \textbf{Axis options:}
\item
\begin{verbatim}
 *axislabcol: color of axis label and numbers. Default is “blue”.*
\end{verbatim}
\item
\begin{verbatim}
 *caxislabels: Character vector to be used as labels on the center axis.*
\end{verbatim}
\end{itemize}

Helper function to produce a beautiful radar chart:

\begin{Shaded}
\begin{Highlighting}[]
\NormalTok{create\_beautiful\_radarchart }\OtherTok{\textless{}{-}} \ControlFlowTok{function}\NormalTok{(data, }\AttributeTok{color =} \StringTok{"\#00AFBB"}\NormalTok{, }
                                        \AttributeTok{vlabels =} \FunctionTok{colnames}\NormalTok{(data), }\AttributeTok{vlcex =} \FloatTok{0.7}\NormalTok{,}
                                        \AttributeTok{caxislabels =} \ConstantTok{NULL}\NormalTok{, }\AttributeTok{title =} \ConstantTok{NULL}\NormalTok{, ...)\{}
  \FunctionTok{radarchart}\NormalTok{(}
\NormalTok{    data, }\AttributeTok{axistype =} \DecValTok{1}\NormalTok{,}
    \CommentTok{\# Customize the polygon}
    \AttributeTok{pcol =}\NormalTok{ color, }\AttributeTok{pfcol =}\NormalTok{ scales}\SpecialCharTok{::}\FunctionTok{alpha}\NormalTok{(color, }\FloatTok{0.5}\NormalTok{), }\AttributeTok{plwd =} \DecValTok{2}\NormalTok{, }\AttributeTok{plty =} \DecValTok{1}\NormalTok{,}
    \CommentTok{\# Customize the grid}
    \AttributeTok{cglcol =} \StringTok{"grey"}\NormalTok{, }\AttributeTok{cglty =} \DecValTok{1}\NormalTok{, }\AttributeTok{cglwd =} \FloatTok{0.8}\NormalTok{,}
    \CommentTok{\# Customize the axis}
    \AttributeTok{axislabcol =} \StringTok{"grey"}\NormalTok{, }
    \CommentTok{\# Variable labels}
    \AttributeTok{vlcex =}\NormalTok{ vlcex, }\AttributeTok{vlabels =}\NormalTok{ vlabels,}
    \AttributeTok{caxislabels =}\NormalTok{ caxislabels, }\AttributeTok{title =}\NormalTok{ title, ...}
\NormalTok{  )}
\NormalTok{\}}
\end{Highlighting}
\end{Shaded}

In the code above, we used the function alpha() {[}in scales package{]}
to change the polygon fill color transparency.

\begin{Shaded}
\begin{Highlighting}[]
\CommentTok{\# Reduce plot margin using par()}
\NormalTok{op }\OtherTok{\textless{}{-}} \FunctionTok{par}\NormalTok{(}\AttributeTok{mar =} \FunctionTok{c}\NormalTok{(}\DecValTok{1}\NormalTok{, }\DecValTok{2}\NormalTok{, }\DecValTok{2}\NormalTok{, }\DecValTok{1}\NormalTok{))}
\FunctionTok{create\_beautiful\_radarchart}\NormalTok{(student1\_data, }\AttributeTok{caxislabels =} \FunctionTok{c}\NormalTok{(}\DecValTok{0}\NormalTok{, }\DecValTok{5}\NormalTok{, }\DecValTok{10}\NormalTok{, }\DecValTok{15}\NormalTok{, }\DecValTok{20}\NormalTok{))}
\end{Highlighting}
\end{Shaded}

\includegraphics{Radar_chart_markdown_files/figure-latex/unnamed-chunk-9-1.pdf}

\begin{Shaded}
\begin{Highlighting}[]
\FunctionTok{par}\NormalTok{(op)}
\end{Highlighting}
\end{Shaded}

\textbf{Create radar charts for multiple individuals}

Create the radar chart of the three students on the same plot:

\begin{Shaded}
\begin{Highlighting}[]
\CommentTok{\# Reduce plot margin using par()}
\NormalTok{op }\OtherTok{\textless{}{-}} \FunctionTok{par}\NormalTok{(}\AttributeTok{mar =} \FunctionTok{c}\NormalTok{(}\DecValTok{1}\NormalTok{, }\DecValTok{2}\NormalTok{, }\DecValTok{2}\NormalTok{, }\DecValTok{2}\NormalTok{))}
\CommentTok{\# Create the radar charts}
\FunctionTok{create\_beautiful\_radarchart}\NormalTok{(}
  \AttributeTok{data =}\NormalTok{ df, }\AttributeTok{caxislabels =} \FunctionTok{c}\NormalTok{(}\DecValTok{0}\NormalTok{, }\DecValTok{5}\NormalTok{, }\DecValTok{10}\NormalTok{, }\DecValTok{15}\NormalTok{, }\DecValTok{20}\NormalTok{),}
  \AttributeTok{color =} \FunctionTok{c}\NormalTok{(}\StringTok{"\#00AFBB"}\NormalTok{, }\StringTok{"\#E7B800"}\NormalTok{, }\StringTok{"\#FC4E07"}\NormalTok{)}
\NormalTok{)}
\CommentTok{\# Add an horizontal legend}
\FunctionTok{legend}\NormalTok{(}
  \AttributeTok{x =} \StringTok{"topright"}\NormalTok{, }\AttributeTok{legend =} \FunctionTok{rownames}\NormalTok{(df[}\SpecialCharTok{{-}}\FunctionTok{c}\NormalTok{(}\DecValTok{1}\NormalTok{,}\DecValTok{2}\NormalTok{),]), }\AttributeTok{horiz =} \ConstantTok{FALSE}\NormalTok{,}
  \AttributeTok{bty =} \StringTok{"n"}\NormalTok{, }\AttributeTok{pch =} \DecValTok{20}\NormalTok{ , }\AttributeTok{col =} \FunctionTok{c}\NormalTok{(}\StringTok{"\#00AFBB"}\NormalTok{, }\StringTok{"\#E7B800"}\NormalTok{, }\StringTok{"\#FC4E07"}\NormalTok{),}
  \AttributeTok{text.col =} \StringTok{"black"}\NormalTok{, }\AttributeTok{cex =} \DecValTok{1}\NormalTok{, }\AttributeTok{pt.cex =} \FloatTok{1.5}
\NormalTok{)}
\end{Highlighting}
\end{Shaded}

\includegraphics{Radar_chart_markdown_files/figure-latex/unnamed-chunk-10-1.pdf}

\begin{Shaded}
\begin{Highlighting}[]
\FunctionTok{par}\NormalTok{(op)}
\end{Highlighting}
\end{Shaded}

In the legend() function of ggplot2, you can specify the position of the
legend using the x and y arguments. Here are some of the possible
positions for the legend:

\begin{itemize}
\tightlist
\item
  ``top'': Places the legend at the top of the plot.
\item
  ``bottom'': Places the legend at the bottom of the plot.
\item
  ``left'': Positions the legend on the left side of the plot.
\item
  ``right'': Positions the legend on the right side of the plot.
\item
  ``topleft'': Positions the legend in the top left corner of the plot.
\item
  ``topright'': Positions the legend in the top right corner of the
  plot.
\item
  ``bottomleft'': Positions the legend in the bottom left corner of the
  plot.
\item
  ``bottomright'': Positions the legend in the bottom right corner of
  the plot.
\end{itemize}

\emph{c(x\_coordinate, y\_coordinate): Allows you to specify custom
coordinates for the legend. You can provide the x and y coordinates as
numeric values.}

\emph{For example, if you want to place the legend on the top right
corner of the plot, you can use legend(x = ``topright''). Similarly, if
you want to specify custom coordinates, you can use legend(x = 0.8, y =
0.2) to position the legend at the coordinates (0.8, 0.2) within the
plot.}

\emph{These different positions give you flexibility in placing the
legend in various parts of the plot depending on your visualization
requirements.}

\textbf{Create separated spider charts for each individual. This is
recommended when you have more than 3 series.}

\begin{Shaded}
\begin{Highlighting}[]
\CommentTok{\# Define colors and titles}

\NormalTok{colors }\OtherTok{\textless{}{-}} \FunctionTok{c}\NormalTok{(}\StringTok{"\#00AFBB"}\NormalTok{, }\StringTok{"\#E7B800"}\NormalTok{, }\StringTok{"\#FC4E07"}\NormalTok{)}
\NormalTok{titles }\OtherTok{\textless{}{-}} \FunctionTok{c}\NormalTok{(}\StringTok{"Student.1"}\NormalTok{, }\StringTok{"Student.2"}\NormalTok{, }\StringTok{"Student.3"}\NormalTok{)}

\CommentTok{\# Reduce plot margin using par()}
\CommentTok{\# Split the screen in 3 parts}
\NormalTok{op }\OtherTok{\textless{}{-}} \FunctionTok{par}\NormalTok{(}\AttributeTok{mar =} \FunctionTok{c}\NormalTok{(}\DecValTok{1}\NormalTok{, }\DecValTok{1}\NormalTok{, }\DecValTok{1}\NormalTok{, }\DecValTok{1}\NormalTok{))}
\FunctionTok{par}\NormalTok{(}\AttributeTok{mfrow =} \FunctionTok{c}\NormalTok{(}\DecValTok{1}\NormalTok{,}\DecValTok{3}\NormalTok{))}

\CommentTok{\# Create the radar chart}
\ControlFlowTok{for}\NormalTok{(i }\ControlFlowTok{in} \DecValTok{1}\SpecialCharTok{:}\DecValTok{3}\NormalTok{)\{}
  \FunctionTok{create\_beautiful\_radarchart}\NormalTok{(}
    \AttributeTok{data =}\NormalTok{ df[}\FunctionTok{c}\NormalTok{(}\DecValTok{1}\NormalTok{, }\DecValTok{2}\NormalTok{, i}\SpecialCharTok{+}\DecValTok{2}\NormalTok{), ], }\AttributeTok{caxislabels =} \FunctionTok{c}\NormalTok{(}\DecValTok{0}\NormalTok{, }\DecValTok{5}\NormalTok{, }\DecValTok{10}\NormalTok{, }\DecValTok{15}\NormalTok{, }\DecValTok{20}\NormalTok{),}
    \AttributeTok{color =}\NormalTok{ colors[i], }\AttributeTok{title =}\NormalTok{ titles[i]}
\NormalTok{  )}
\NormalTok{\}}
\end{Highlighting}
\end{Shaded}

\includegraphics{Radar_chart_markdown_files/figure-latex/unnamed-chunk-11-1.pdf}

\begin{Shaded}
\begin{Highlighting}[]
\FunctionTok{par}\NormalTok{(op)}
\end{Highlighting}
\end{Shaded}

\textbf{Compare every profile to an average profile}

Radar charts are most useful if the profile of every individual is
compared to an average profile.

\begin{Shaded}
\begin{Highlighting}[]
\CommentTok{\# Create a demo data containing exam scores for 10 students:}

\FunctionTok{set.seed}\NormalTok{(}\DecValTok{123}\NormalTok{)}
\NormalTok{df }\OtherTok{\textless{}{-}} \FunctionTok{as.data.frame}\NormalTok{(}
  \FunctionTok{matrix}\NormalTok{(}\FunctionTok{sample}\NormalTok{(}\DecValTok{2}\SpecialCharTok{:}\DecValTok{20}\NormalTok{ , }\DecValTok{90}\NormalTok{ , }\AttributeTok{replace =} \ConstantTok{TRUE}\NormalTok{),}
         \AttributeTok{ncol=}\DecValTok{9}\NormalTok{, }\AttributeTok{byrow =} \ConstantTok{TRUE}\NormalTok{)}
\NormalTok{)}

\CommentTok{\# In this example the matrix we build up has a vote (variable) between 2 and 20,}
\CommentTok{\# x90 times (total votes=values 10 student x9 subjects), replace the value in }
\CommentTok{\# the empty matrix. ncol = at the number of subject (9), order the frame by }
\CommentTok{\# number of rows of the data frame = 10!}
\NormalTok{df }
\end{Highlighting}
\end{Shaded}

\begin{verbatim}
##    V1 V2 V3 V4 V5 V6 V7 V8 V9
## 1  16 20 15  4 11 19 12  6 15
## 2   6 20 10  4  9  8 11 10 20
## 3   5 15 18 12  8 13 16 11 14
## 4   8 10 10 11  8  7  3  6  9
## 5  13 14 19  2  7 16 10 16 17
## 6   7 12  9  8 17 18 19 18  3
## 7   5 14  6 20 15  4  9 17 13
## 8  15  4 15  8  4 16  6  9 20
## 9  11 19 11 13  3 11 13 15 18
## 10 15  4  9 15 20 16 18 12  8
\end{verbatim}

\begin{Shaded}
\begin{Highlighting}[]
\FunctionTok{colnames}\NormalTok{(df) }\OtherTok{\textless{}{-}} \FunctionTok{c}\NormalTok{(}
  \StringTok{"Biology"}\NormalTok{, }\StringTok{"Physics"}\NormalTok{, }\StringTok{"Maths"}\NormalTok{, }\StringTok{"Sport"}\NormalTok{, }\StringTok{"English"}\NormalTok{, }
  \StringTok{"Geography"}\NormalTok{, }\StringTok{"Art"}\NormalTok{, }\StringTok{"Programming"}\NormalTok{, }\StringTok{"Music"}
\NormalTok{)}
\FunctionTok{rownames}\NormalTok{(df) }\OtherTok{\textless{}{-}} \FunctionTok{paste0}\NormalTok{(}\StringTok{"Student."}\NormalTok{, }\DecValTok{1}\SpecialCharTok{:}\FunctionTok{nrow}\NormalTok{(df))}
\FunctionTok{head}\NormalTok{(df)}
\end{Highlighting}
\end{Shaded}

\begin{verbatim}
##           Biology Physics Maths Sport English Geography Art Programming Music
## Student.1      16      20    15     4      11        19  12           6    15
## Student.2       6      20    10     4       9         8  11          10    20
## Student.3       5      15    18    12       8        13  16          11    14
## Student.4       8      10    10    11       8         7   3           6     9
## Student.5      13      14    19     2       7        16  10          16    17
## Student.6       7      12     9     8      17        18  19          18     3
\end{verbatim}

\begin{Shaded}
\begin{Highlighting}[]
\CommentTok{\# Give the name based on a root = Student and adding the row number from 1 to }
\CommentTok{\# end (10), Same is giving the name of the subject per every column}
\end{Highlighting}
\end{Shaded}

\textbf{Rescale each variable to range between 0 and 1:}

The provided code performs the following operations:

\begin{itemize}
\item
  df\_scaled \textless- round(apply(df, 2, scales::rescale),
  2):\emph{This line scales the columns of the data frame df using the
  scales::rescale function from the scales package. The apply() function
  is used to apply the scaling operation column-wise (2 indicates
  columns) on the data frame. The rescale function scales the values of
  each column to a new range (usually between 0 and 1). The resulting
  scaled values are then rounded to two decimal places using the round()
  function. The scaled values are assigned to a new data frame called
  df\_scaled.}
\item
  df\_scaled \textless- as.data.frame(df\_scaled): \emph{This line
  converts the df\_scaled object (which was previously a matrix
  resulting from the apply function) into a data frame. This step is
  done to ensure that the resulting object is in the desired data frame
  format.}
\item
  head(df\_scaled): \emph{This line displays the first few rows of the
  df\_scaled data frame, allowing you to inspect the scaled values.}
\end{itemize}

In summary, the code scales the columns of the original data frame df
using the rescale function, rounds the scaled values, converts the
result into a data frame, and then displays the first few rows of the
scaled data frame. This is often done to normalize or standardize the
values in a dataset, making them comparable or easier to interpret.

\begin{Shaded}
\begin{Highlighting}[]
\FunctionTok{library}\NormalTok{(scales)}
\NormalTok{df\_scaled }\OtherTok{\textless{}{-}} \FunctionTok{round}\NormalTok{(}\FunctionTok{apply}\NormalTok{(df, }\DecValTok{2}\NormalTok{, scales}\SpecialCharTok{::}\NormalTok{rescale), }\DecValTok{2}\NormalTok{)}
\NormalTok{df\_scaled }\OtherTok{\textless{}{-}} \FunctionTok{as.data.frame}\NormalTok{(df\_scaled)}
\FunctionTok{head}\NormalTok{(df\_scaled)}
\end{Highlighting}
\end{Shaded}

\begin{verbatim}
##           Biology Physics Maths Sport English Geography  Art Programming Music
## Student.1    1.00    1.00  0.69  0.11    0.47      1.00 0.56        0.00  0.71
## Student.2    0.09    1.00  0.31  0.11    0.35      0.27 0.50        0.33  1.00
## Student.3    0.00    0.69  0.92  0.56    0.29      0.60 0.81        0.42  0.65
## Student.4    0.27    0.38  0.31  0.50    0.29      0.20 0.00        0.00  0.35
## Student.5    0.73    0.62  1.00  0.00    0.24      0.80 0.44        0.83  0.82
## Student.6    0.18    0.50  0.23  0.33    0.82      0.93 1.00        1.00  0.00
\end{verbatim}

To scale the data between 0 and 9 instead of the default range (usually
0 to 1), you can modify the code as follows:

\emph{df\_scaled \textless- round(apply(df, 2, function(x)
scales::rescale(x, to = c(0, 9))), 2)} \emph{df\_scaled \textless-
as.data.frame(df\_scaled)} \emph{head(df\_scaled)}
--------------------------------------------------------------------------------

\textbf{Prepare the data for creating the radar plot using the fmsb
package:}

\begin{Shaded}
\begin{Highlighting}[]
\CommentTok{\# Variables summary}
\CommentTok{\# Get the minimum and the max of every column  }
\NormalTok{col\_max }\OtherTok{\textless{}{-}} \FunctionTok{apply}\NormalTok{(df\_scaled, }\DecValTok{2}\NormalTok{, max)}
\NormalTok{col\_min }\OtherTok{\textless{}{-}} \FunctionTok{apply}\NormalTok{(df\_scaled, }\DecValTok{2}\NormalTok{, min)}
\CommentTok{\# Calculate the average profile }
\NormalTok{col\_mean }\OtherTok{\textless{}{-}} \FunctionTok{apply}\NormalTok{(df\_scaled, }\DecValTok{2}\NormalTok{, mean)}
\CommentTok{\# Put together the summary of columns}
\NormalTok{col\_summary }\OtherTok{\textless{}{-}} \FunctionTok{t}\NormalTok{(}\FunctionTok{data.frame}\NormalTok{(}\AttributeTok{Max =}\NormalTok{ col\_max, }\AttributeTok{Min =}\NormalTok{ col\_min, }\AttributeTok{Average =}\NormalTok{ col\_mean))}


\CommentTok{\# Bind variables summary to the data}
\NormalTok{df\_scaled2 }\OtherTok{\textless{}{-}} \FunctionTok{as.data.frame}\NormalTok{(}\FunctionTok{rbind}\NormalTok{(col\_summary, df\_scaled))}
\FunctionTok{head}\NormalTok{(df\_scaled2)}
\end{Highlighting}
\end{Shaded}

\begin{verbatim}
##           Biology Physics Maths Sport English Geography   Art Programming Music
## Max         1.000   1.000 1.000 1.000   1.000     1.000 1.000        1.00 1.000
## Min         0.000   0.000 0.000 0.000   0.000     0.000 0.000        0.00 0.000
## Average     0.464   0.575 0.476 0.427   0.423     0.587 0.544        0.50 0.629
## Student.1   1.000   1.000 0.690 0.110   0.470     1.000 0.560        0.00 0.710
## Student.2   0.090   1.000 0.310 0.110   0.350     0.270 0.500        0.33 1.000
## Student.3   0.000   0.690 0.920 0.560   0.290     0.600 0.810        0.42 0.650
\end{verbatim}

Here we created a data frame in which are present also the minimum
scaled value, the maximum scaled value and the mean scaled value, for
every subject in our matrix. This is added to the original data frame
above the others rows.

\textbf{Produce radar plots showing both the average profile and the
individual profile:}

\begin{Shaded}
\begin{Highlighting}[]
\NormalTok{opar }\OtherTok{\textless{}{-}} \FunctionTok{par}\NormalTok{() }
\CommentTok{\# Define settings for plotting in a 3x4 grid, with appropriate margins:}
\FunctionTok{par}\NormalTok{(}\AttributeTok{mar =} \FunctionTok{rep}\NormalTok{(}\FloatTok{0.8}\NormalTok{,}\DecValTok{4}\NormalTok{))}
\FunctionTok{par}\NormalTok{(}\AttributeTok{mfrow =} \FunctionTok{c}\NormalTok{(}\DecValTok{3}\NormalTok{,}\DecValTok{4}\NormalTok{))}
\CommentTok{\# Produce a radar{-}chart for each student}
\ControlFlowTok{for}\NormalTok{ (i }\ControlFlowTok{in} \DecValTok{4}\SpecialCharTok{:}\FunctionTok{nrow}\NormalTok{(df\_scaled2)) \{}
  \FunctionTok{radarchart}\NormalTok{(}
\NormalTok{    df\_scaled2[}\FunctionTok{c}\NormalTok{(}\DecValTok{1}\SpecialCharTok{:}\DecValTok{3}\NormalTok{, i), ],}
    \AttributeTok{pfcol =} \FunctionTok{c}\NormalTok{(}\StringTok{"\#99999980"}\NormalTok{,}\ConstantTok{NA}\NormalTok{),}
    \AttributeTok{pcol=} \FunctionTok{c}\NormalTok{(}\ConstantTok{NA}\NormalTok{,}\DecValTok{2}\NormalTok{), }\AttributeTok{plty =} \DecValTok{1}\NormalTok{, }\AttributeTok{plwd =} \DecValTok{2}\NormalTok{,}
    \AttributeTok{title =} \FunctionTok{row.names}\NormalTok{(df\_scaled2)[i]}
\NormalTok{  )}
\NormalTok{\}}
\CommentTok{\# Restore the standard par() settings}
\NormalTok{par }\OtherTok{\textless{}{-}} \FunctionTok{par}\NormalTok{(opar) }
\end{Highlighting}
\end{Shaded}

\includegraphics{Radar_chart_markdown_files/figure-latex/unnamed-chunk-15-1.pdf}

The provided code performs the following operations:

\begin{itemize}
\item
  \begin{enumerate}
  \def\labelenumi{\arabic{enumi}.}
  \tightlist
  \item
    \texttt{opar\ \textless{}-\ par()}: This line saves the current
    graphical parameters in the \texttt{opar} object. It is done so that
    you can later restore the original settings after making changes.
  \end{enumerate}
\item
  \begin{enumerate}
  \def\labelenumi{\arabic{enumi}.}
  \setcounter{enumi}{1}
  \tightlist
  \item
    \texttt{par(mar\ =\ rep(0.8,\ 4))}: This sets the margins (in
    inches) for the plot. By using \texttt{rep(0.8,\ 4)}, it sets all
    four margins (bottom, left, top, right) to a value of 0.8 inches.
    This adjusts the space between the plot area and the edges of the
    graphics device.
  \end{enumerate}
\item
  \begin{enumerate}
  \def\labelenumi{\arabic{enumi}.}
  \setcounter{enumi}{2}
  \tightlist
  \item
    \texttt{par(mfrow\ =\ c(3,\ 4))}: This sets the layout of the plots
    in a 3x4 grid. It divides the graphics device into a grid with 3
    rows and 4 columns, allowing you to create multiple plots in a
    single figure.
  \end{enumerate}
\item
  \begin{enumerate}
  \def\labelenumi{\arabic{enumi}.}
  \setcounter{enumi}{3}
  \tightlist
  \item
    The \texttt{for} loop: This loop iterates over the rows of the
    \texttt{df\_scaled2} data frame, starting from the 4th row
    (\texttt{4:nrow(df\_scaled2)}). It produces a radar chart for each
    student using the \texttt{radarchart()} function.
  \end{enumerate}
\item
  \begin{enumerate}
  \def\labelenumi{\arabic{enumi}.}
  \setcounter{enumi}{4}
  \tightlist
  \item
    Inside the loop, \texttt{radarchart()} function is called with the
    following arguments:
  \end{enumerate}

  \begin{itemize}
  \tightlist
  \item
    \texttt{df\_scaled2{[}c(1:3,\ i),\ {]}}: This selects the rows 1 to
    3 and the current \texttt{i}th row from the \texttt{df\_scaled2}
    data frame, creating a subset of the data to be plotted.
  \item
    \texttt{pfcol\ =\ c("\#99999980",\ NA)}: This sets the fill color
    for the polygon area in the radar chart. The first color
    \texttt{\#99999980} represents a light gray color with 50\%
    transparency, and \texttt{NA} indicates no fill color for the data
    points.
  \item
    \texttt{pcol\ =\ c(NA,\ 2)}: This sets the color for the polygon
    border and the data points. \texttt{NA} means no border color for
    the polygon, and \texttt{2} represents a color code for the data
    points.
  \item
    \texttt{plty\ =\ 1}: This sets the line type for the polygon border
    to a solid line.
  \item
    \texttt{plwd\ =\ 2}: This sets the line width for the polygon border
    to a value of 2.
  \item
    \texttt{title\ =\ row.names(df\_scaled2){[}i{]}}: This sets the
    title of the radar chart to the row name of the current \texttt{i}th
    row in the \texttt{df\_scaled2} data frame.
  \end{itemize}
\item
  \begin{enumerate}
  \def\labelenumi{\arabic{enumi}.}
  \setcounter{enumi}{5}
  \tightlist
  \item
    \texttt{par\ \textless{}-\ par(opar)}: This line restores the
    original graphical parameters by assigning the saved \texttt{opar}
    object back to \texttt{par}. It ensures that the settings are
    reverted to the state before the changes made in the code.
  \end{enumerate}
\end{itemize}

Overall, this code generates a grid of radar charts, with each chart
representing a different student's data. The charts are created using
the \texttt{radarchart()} function and customized using various
parameters. The \texttt{par()} function is used to modify and restore
graphical settings for the plot.

================================================================================
\# Forth Part

\textbf{ggplot radar chart using the ggradar R package}

Prerequisites

\begin{Shaded}
\begin{Highlighting}[]
\FunctionTok{library}\NormalTok{ (ggradar)}
\end{Highlighting}
\end{Shaded}

Key function and arguments

\texttt{ggradar(plot.data,\ values.radar\ =\ c("0\%",\ "50\%",\ "100\%"),grid.min\ =\ 0,\ grid.mid\ =\ 0.5,\ grid.max\ =\ 1,\ )}

\begin{itemize}
\tightlist
\item
  plot.data: data containing one row per individual or group
\item
  values.radar: values to show at minimum, average and maximum grid
  lines
\item
  grid.min: value at which minimum grid line is plotted
\item
  grid.mid: value at which average grid line is plotted
\item
  grid.max: value at which maximum grid line is plotted
\end{itemize}

\textbf{Data preparation}

NB:

All variables in the data should be at the same scale. If this is not
the case, you need to rescale the data.

For example, you can rescale the variables to have a minimum of 0 and a
maximum of 1 using the function \texttt{rescale()} {[}scales package{]}.
We'll describe this method in the next sections.

\begin{Shaded}
\begin{Highlighting}[]
\CommentTok{\# Put row names into  a column named group}
\FunctionTok{library}\NormalTok{(tidyverse)}
\NormalTok{df }\OtherTok{\textless{}{-}}\NormalTok{ exam\_scores }\SpecialCharTok{\%\textgreater{}\%} \FunctionTok{rownames\_to\_column}\NormalTok{(}\StringTok{"group"}\NormalTok{)}
\NormalTok{df}
\end{Highlighting}
\end{Shaded}

\begin{verbatim}
##       group Biology Physics Maths Sport English Geography Art Programming Music
## 1 Student.1     7.9      10   3.7   8.7     7.9       6.4 2.4           0    20
## 2 Student.2     3.9      20  11.5  20.0     7.2      10.5 0.2           0    20
## 3 Student.3     9.4       0   2.5   4.0    12.4       6.5 9.8          20    20
\end{verbatim}

\textbf{Basic radar plot}

\begin{Shaded}
\begin{Highlighting}[]
\CommentTok{\# Plotting student 1}
\FunctionTok{ggradar}\NormalTok{(}
\NormalTok{  df[}\DecValTok{1}\NormalTok{, ], }
  \AttributeTok{values.radar =} \FunctionTok{c}\NormalTok{(}\StringTok{"0"}\NormalTok{, }\StringTok{"10"}\NormalTok{, }\StringTok{"20"}\NormalTok{),}
  \AttributeTok{grid.min =} \DecValTok{0}\NormalTok{, }\AttributeTok{grid.mid =} \DecValTok{10}\NormalTok{, }\AttributeTok{grid.max =} \DecValTok{20}
\NormalTok{)}
\end{Highlighting}
\end{Shaded}

\includegraphics{Radar_chart_markdown_files/figure-latex/unnamed-chunk-18-1.pdf}

\textbf{Customize radar charts}

Key arguments to customize the different components of the ggplot radar
chart. For more options see the documentation.

\begin{Shaded}
\begin{Highlighting}[]
\FunctionTok{ggradar}\NormalTok{(}
\NormalTok{  df[}\DecValTok{1}\NormalTok{, ], }
  \AttributeTok{values.radar =} \FunctionTok{c}\NormalTok{(}\StringTok{"0"}\NormalTok{, }\StringTok{"10"}\NormalTok{, }\StringTok{"20"}\NormalTok{),}
  \AttributeTok{grid.min =} \DecValTok{0}\NormalTok{, }\AttributeTok{grid.mid =} \DecValTok{10}\NormalTok{, }\AttributeTok{grid.max =} \DecValTok{20}\NormalTok{,}
  \CommentTok{\# Polygons}
  \AttributeTok{group.line.width =} \DecValTok{1}\NormalTok{, }
  \AttributeTok{group.point.size =} \DecValTok{3}\NormalTok{,}
  \AttributeTok{group.colours =} \StringTok{"\#00AFBB"}\NormalTok{,}
  \CommentTok{\# Background and grid lines}
  \AttributeTok{background.circle.colour =} \StringTok{"white"}\NormalTok{,}
  \AttributeTok{gridline.mid.colour =} \StringTok{"grey"}
\NormalTok{  )}
\end{Highlighting}
\end{Shaded}

\includegraphics{Radar_chart_markdown_files/figure-latex/unnamed-chunk-19-1.pdf}

\textbf{Radar chart with multiple individuals or groups}

Create the radar chart of the three students on the same plot:

\begin{Shaded}
\begin{Highlighting}[]
\FunctionTok{ggradar}\NormalTok{(}
\NormalTok{  df, }
  \AttributeTok{values.radar =} \FunctionTok{c}\NormalTok{(}\StringTok{"0"}\NormalTok{, }\StringTok{"10"}\NormalTok{, }\StringTok{"20"}\NormalTok{),}
  \AttributeTok{grid.min =} \DecValTok{0}\NormalTok{, }\AttributeTok{grid.mid =} \DecValTok{10}\NormalTok{, }\AttributeTok{grid.max =} \DecValTok{20}\NormalTok{,}
  \CommentTok{\# Polygons}
  \AttributeTok{group.line.width =} \DecValTok{1}\NormalTok{, }
  \AttributeTok{group.point.size =} \DecValTok{3}\NormalTok{,}
  \AttributeTok{group.colours =} \FunctionTok{c}\NormalTok{(}\StringTok{"\#00AFBB"}\NormalTok{, }\StringTok{"\#E7B800"}\NormalTok{, }\StringTok{"\#FC4E07"}\NormalTok{),}
  \CommentTok{\# Background and grid lines}
  \AttributeTok{background.circle.colour =} \StringTok{"white"}\NormalTok{,}
  \AttributeTok{gridline.mid.colour =} \StringTok{"grey"}\NormalTok{,}
  \AttributeTok{legend.position =} \StringTok{"bottom"}
\NormalTok{  )}
\end{Highlighting}
\end{Shaded}

\includegraphics{Radar_chart_markdown_files/figure-latex/unnamed-chunk-20-1.pdf}

\textbf{Alternatives to radar charts}

A circular plot is difficult to read. An alternative to a radar chart is
an ordered lolliplot or dotchart. This section describes how to create
dotcharts. The ggpubr R package will be used in this section to create a
dotchart.

Load required packages:

\begin{Shaded}
\begin{Highlighting}[]
\FunctionTok{library}\NormalTok{(tidyverse)}
\FunctionTok{library}\NormalTok{(ggpubr)}
\end{Highlighting}
\end{Shaded}

\textbf{Case when all quantitative variables have the same scale}

Displaying one individual

Data preparation:

\begin{Shaded}
\begin{Highlighting}[]
\NormalTok{df2 }\OtherTok{\textless{}{-}} \FunctionTok{t}\NormalTok{(exam\_scores) }\SpecialCharTok{\%\textgreater{}\%}
  \FunctionTok{as.data.frame}\NormalTok{() }\SpecialCharTok{\%\textgreater{}\%}
  \FunctionTok{rownames\_to\_column}\NormalTok{(}\StringTok{"Field"}\NormalTok{)}
\NormalTok{df2}
\end{Highlighting}
\end{Shaded}

\begin{verbatim}
##         Field Student.1 Student.2 Student.3
## 1     Biology       7.9       3.9       9.4
## 2     Physics      10.0      20.0       0.0
## 3       Maths       3.7      11.5       2.5
## 4       Sport       8.7      20.0       4.0
## 5     English       7.9       7.2      12.4
## 6   Geography       6.4      10.5       6.5
## 7         Art       2.4       0.2       9.8
## 8 Programming       0.0       0.0      20.0
## 9       Music      20.0      20.0      20.0
\end{verbatim}

The code provided performs the following operations:

\begin{itemize}
\tightlist
\item
  \textbf{t(exam\_scores)}: This transposes the exam\_scores object. If
  exam\_scores is a matrix or data frame, transposing it will swap the
  rows and columns, resulting in a new object.
\end{itemize}

\%\textgreater\%: This is the pipe operator, which allows you to chain
multiple operations together. It takes the output from the previous
operation and passes it as the first argument to the next function call.

\begin{itemize}
\item
  \textbf{as.data.frame()}: This converts the transposed object into a
  data frame. If exam\_scores was originally a matrix, this step
  converts it into a data frame format.
\item
  \textbf{rownames\_to\_column(``Field'')}: This function from the dplyr
  package adds a new column called ``Field'' to the data frame,
  containing the row names from the transposed object. It assigns the
  row names as a separate column, making them accessible for further
  data manipulation or analysis.
\item
  \textbf{df2}: This assigns the resulting object from the previous
  operations to the variable df2.
\end{itemize}

Overall, this code takes the exam\_scores object, transposes it,
converts it into a data frame, and adds a new column with the row names.
The final result is stored in the df2 variable, which contains the
transposed data with row names as a separate column.

\textbf{Plot creation:}

\begin{Shaded}
\begin{Highlighting}[]
\FunctionTok{ggdotchart}\NormalTok{(}
\NormalTok{  df2, }\AttributeTok{x =} \StringTok{"Field"}\NormalTok{, }\AttributeTok{y =} \StringTok{"Student.1"}\NormalTok{,}
  \AttributeTok{add =} \StringTok{"segments"}\NormalTok{, }\AttributeTok{sorting =} \StringTok{"descending"}\NormalTok{,}
  \AttributeTok{ylab =} \StringTok{"Exam Score"}\NormalTok{, }\AttributeTok{title =} \StringTok{"Student 1"}
\NormalTok{  )}
\end{Highlighting}
\end{Shaded}

\includegraphics{Radar_chart_markdown_files/figure-latex/unnamed-chunk-23-1.pdf}

\textbf{Displaying two individuals}

Data preparation:

\begin{Shaded}
\begin{Highlighting}[]
\NormalTok{df3 }\OtherTok{\textless{}{-}}\NormalTok{ df2 }\SpecialCharTok{\%\textgreater{}\%}
  \FunctionTok{select}\NormalTok{(Field, Student}\FloatTok{.1}\NormalTok{, Student}\FloatTok{.2}\NormalTok{) }\SpecialCharTok{\%\textgreater{}\%}
  \FunctionTok{pivot\_longer}\NormalTok{(}
    \AttributeTok{cols =} \FunctionTok{c}\NormalTok{(Student}\FloatTok{.1}\NormalTok{, Student}\FloatTok{.2}\NormalTok{),}
    \AttributeTok{names\_to =} \StringTok{"student"}\NormalTok{,}
    \AttributeTok{values\_to =} \StringTok{"value"}
\NormalTok{  )}
\FunctionTok{head}\NormalTok{(df3)}
\end{Highlighting}
\end{Shaded}

\begin{verbatim}
## # A tibble: 6 x 3
##   Field   student   value
##   <chr>   <chr>     <dbl>
## 1 Biology Student.1   7.9
## 2 Biology Student.2   3.9
## 3 Physics Student.1  10  
## 4 Physics Student.2  20  
## 5 Maths   Student.1   3.7
## 6 Maths   Student.2  11.5
\end{verbatim}

\textbf{Plot creation}

\begin{Shaded}
\begin{Highlighting}[]
\FunctionTok{ggdotchart}\NormalTok{(}
\NormalTok{  df3, }\AttributeTok{x =} \StringTok{"Field"}\NormalTok{, }\AttributeTok{y =} \StringTok{"value"}\NormalTok{, }
  \AttributeTok{group =} \StringTok{"student"}\NormalTok{, }\AttributeTok{color =} \StringTok{"student"}\NormalTok{, }\AttributeTok{palette =} \StringTok{"jco"}\NormalTok{,}
  \AttributeTok{add =} \StringTok{"segment"}\NormalTok{, }\AttributeTok{position =} \FunctionTok{position\_dodge}\NormalTok{(}\FloatTok{0.3}\NormalTok{),}
  \AttributeTok{sorting =} \StringTok{"descending"}
\NormalTok{  )}
\end{Highlighting}
\end{Shaded}

\includegraphics{Radar_chart_markdown_files/figure-latex/unnamed-chunk-25-1.pdf}

\textbf{Displaying multiple individuals}

Data preparation:

\begin{Shaded}
\begin{Highlighting}[]
\NormalTok{df4 }\OtherTok{\textless{}{-}}\NormalTok{ df2 }\SpecialCharTok{\%\textgreater{}\%}
  \FunctionTok{select}\NormalTok{(Field, Student}\FloatTok{.1}\NormalTok{, Student}\FloatTok{.2}\NormalTok{, Student}\FloatTok{.3}\NormalTok{) }\SpecialCharTok{\%\textgreater{}\%}
  \FunctionTok{pivot\_longer}\NormalTok{(}
    \AttributeTok{cols =} \FunctionTok{c}\NormalTok{(Student}\FloatTok{.1}\NormalTok{, Student}\FloatTok{.2}\NormalTok{, Student}\FloatTok{.3}\NormalTok{),}
    \AttributeTok{names\_to =} \StringTok{"student"}\NormalTok{,}
    \AttributeTok{values\_to =} \StringTok{"value"}
\NormalTok{  )}
\FunctionTok{head}\NormalTok{(df4)}
\end{Highlighting}
\end{Shaded}

\begin{verbatim}
## # A tibble: 6 x 3
##   Field   student   value
##   <chr>   <chr>     <dbl>
## 1 Biology Student.1   7.9
## 2 Biology Student.2   3.9
## 3 Biology Student.3   9.4
## 4 Physics Student.1  10  
## 5 Physics Student.2  20  
## 6 Physics Student.3   0
\end{verbatim}

\textbf{Plot creation}

\begin{Shaded}
\begin{Highlighting}[]
\FunctionTok{ggdotchart}\NormalTok{(}
\NormalTok{  df4, }\AttributeTok{x =} \StringTok{"Field"}\NormalTok{, }\AttributeTok{y =} \StringTok{"value"}\NormalTok{, }
  \AttributeTok{group =} \StringTok{"student"}\NormalTok{, }\AttributeTok{color =} \StringTok{"student"}\NormalTok{, }\AttributeTok{palette =} \StringTok{"jco"}\NormalTok{,}
  \AttributeTok{add =} \StringTok{"segment"}\NormalTok{, }\AttributeTok{position =} \FunctionTok{position\_dodge}\NormalTok{(}\FloatTok{0.3}\NormalTok{),}
  \AttributeTok{sorting =} \StringTok{"descending"}\NormalTok{, }\AttributeTok{facet.by =} \StringTok{"student"}\NormalTok{,}
  \AttributeTok{rotate =} \ConstantTok{TRUE}\NormalTok{, }\AttributeTok{legend =} \StringTok{"none"}
\NormalTok{  )}
\end{Highlighting}
\end{Shaded}

\includegraphics{Radar_chart_markdown_files/figure-latex/unnamed-chunk-27-1.pdf}

\hypertarget{case-when-you-have-a-lot-of-individuals-to-plot-or-if-your-variables-have-different-scales}{%
\section{Case when you have a lot of individuals to plot or if your
variables have different
scales}\label{case-when-you-have-a-lot-of-individuals-to-plot-or-if-your-variables-have-different-scales}}

\textbf{A solution is to create a parallel coordinates plot.}

\begin{Shaded}
\begin{Highlighting}[]
\FunctionTok{library}\NormalTok{(GGally)}

\FunctionTok{ggparcoord}\NormalTok{(}
\NormalTok{  iris,}
  \AttributeTok{columns =} \DecValTok{1}\SpecialCharTok{:}\DecValTok{4}\NormalTok{, }\AttributeTok{groupColumn =} \DecValTok{5}\NormalTok{, }\AttributeTok{order =} \StringTok{"anyClass"}\NormalTok{,}
  \AttributeTok{showPoints =} \ConstantTok{TRUE}\NormalTok{, }
  \AttributeTok{title =} \StringTok{"Parallel Coordinate Plot for the Iris Data"}\NormalTok{,}
  \AttributeTok{alphaLines =} \FloatTok{0.3}
\NormalTok{) }\SpecialCharTok{+} 
  \FunctionTok{theme\_bw}\NormalTok{() }\SpecialCharTok{+}
  \FunctionTok{theme}\NormalTok{(}\AttributeTok{legend.position =} \StringTok{"top"}\NormalTok{)}
\end{Highlighting}
\end{Shaded}

\includegraphics{Radar_chart_markdown_files/figure-latex/unnamed-chunk-28-1.pdf}

Note that, the default of the function ggparcoord() is to rescale each
variable by subtracting the mean and dividing by the standard deviation.

\begin{center}\rule{0.5\linewidth}{0.5pt}\end{center}

. \textbf{END} .

\begin{center}\rule{0.5\linewidth}{0.5pt}\end{center}

\begin{Shaded}
\begin{Highlighting}[]
\FunctionTok{sessionInfo}\NormalTok{()}
\end{Highlighting}
\end{Shaded}


\end{document}
